\section{Scintillator Materials}\label{scintillator-materials}

\subsection{History of Scintillators and the Nonlinearity
Problem}\label{history-of-scintillators-and-the-nonlinearity-problem}

\subsection{Why Sodium Iodide}\label{why-sodium-iodide}

Sodium Iodide is a simple, cubic crystal that is easy to grow in large
quantities. (CITATION + discussion more in depth). It was one of the
first materials to be shown to have scintillating properties when
(CITATION + STORY). Because it has been so heavily studied and because
it is easy to produce, sodium iodide was used in many of the first
radiation portal monitor devices commissioned by the government.

Unfortunately, sodium iodide is also one of the most
\emph{non-proportional} scintillating materials we know about
\ref{Fig:propotionality_graph}. This has lead to the problem of high
false alarm rates among the RPM used in the United States. While many
properties are known about sodium iodide, Auger recombination has proven
difficult to understand because of the aforementioned theoretical and
experimental hurdles. To our (my?) knowledge, there is no theoretical
studies of sodium iodide.

Experimental measurements of Auger recombination in sodium iodide have
widely varying results. ::Review literature on Sodium Iodide here. The
following are the papers you cite in paper::

\begin{itemize}
\tightlist
\item
  G. Bizarri, N. Cherepy, W. S. Choong, G. Hull, W. Moses, S. Payne, J.
  Singh, J. Valentine, A. N. Vasilev, and R. Williams, IEEE Trans. Nucl.
  Sci. 56, 2313 (2009).
\item
  W. W. Moses, S. A. Payne, W.-S. Choong, G. Hull, and B. W. Reutter,
  IEEE Trans. Nucl. Sci. 55, 1049 (2008).
\item
  S. Payne, W. W. Moses, S. Sheets, L. Ahle, N. Cherepy, B. Sturm, S.
  Dazeley, G. Bizarri, and W.-S. Choong, IEEE Trans. Nucl. Sci. 58, 3392
  (2011).
\end{itemize}

(KMC studies on this?)

(Discussion of expectations with phonon-assisted vs.~direct?)

Because of the need for accurate Auger recombination numbers for KMC
modeling by (GROUP) and because of the large variance among experimental
measurements, sodium iodide is an ideal candidate for study using our
first-principles methods.

\subsection{Auger Recombination in Sodium
Iodide}\label{auger-recombination-in-sodium-iodide}

\subsection{Future Work}\label{future-work}
