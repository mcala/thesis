% pandoc-xnos: cleveref formatting
\crefformat{figure}{figure~#2#1#3}
\Crefformat{figure}{Figure~#2#1#3}

\section{Auger Recombination Code Development}\label{chapter:code}

\subsection{Theory of Wannier
FUnctions}\label{theory-of-wannier-functions}

The idea behind Wannier functions is to take the extended, Bloch like
states that we use in our first principles code and reduce them to
localized states in real space. You can loosely think about this in
terms of Fourier transforms between wave functions in k and real space.
In k space, wave functions are spread out over entire supercells which
will correspond to localized wave functions in real space. The question
is how to actually find these analytically.

We run into two problems with the arbitrary nature of Wannier functions.
The first is the familiar phase factor that can be added to any wave
funciton. The second is the Gauge choice of specifically defining the
Bloch functions. This is slightly different than the phase choice, and
leads Wannier functions to be doubly arbitrary. So what method can we
use to actually choose them and to make sure that they are localized.

One method for choosing the gauge is to make sure that the gauge is
\emph{smooth}. That is, the wave function doesn't jump around all that
much. You might think that the Bloch wave functions are going to be
smooth, but because of crossings and degeneracies they probably are not
going to be. In these cases, we can define Unitary matricies that get
rid of those discontinuities and apply them to our Bloch Functions.
These may no longer be eigenvalues of the Hamiltonian, but that's okay,
we can always transform back eventually. These new Bloch functions have
a smooth gauge, which will lead to localized Wannier Functions.

But that's only part of the story. We still need some sort of
minimization condition to make sure that they are localized. So what we
do is define a localization minimization condition involving the
position operator. From here, we rewrite that in terms of the Bloch
functions and figure out how to actually minimize this condition.

When the bands are entangled there are some special considerations that
we need to go through.

Local Minima

Reason why this method for Wannier is the best one?

Discussion of Bloch's Theorem

Discussion of Fourier Transforms \ldots{}luckily, the Fourier transform
is unitary and therefore will preserve the norms of the Hilbert Space.

\subsection{Approximations for Tractable
Calculations}\label{approximations-for-tractable-calculations}

\subsection{Refactoring and
Optimization}\label{refactoring-and-optimization}

\subsection{Parallelism and
Optimization}\label{parallelism-and-optimization}

\subsection{Future Work}\label{future-work}
